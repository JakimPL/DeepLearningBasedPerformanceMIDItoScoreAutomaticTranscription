\begin{tikzpicture}

\coordinate (origin) at (0,0);
\coordinate (stave) at (origin);
% left line of first key
\draw (9.25,-1) -- (9.25,-5);

\def\midiC{0}
\def\midiD{2}
\def\midiE{4}
\def\midiF{5}
\def\midiG{7}
\def\midiA{9}
\def\midiB{11}

% Define a command to calculate the MIDI note number
\newcommand{\midiNote}[3]{% #1 is pitch, #2 is octave
    % Assign MIDI note numbers to each pitch (C=0, C#=1, D=2, ..., B=11)
    \pgfmathtruncatemacro{\pitchValue}{\csname midi#1\endcsname}
    \pgfmathtruncatemacro{\midiNumber}{12 + \pitchValue + #2 * 12 + #3}
};

\newcommand{\drawPianoKey}[4]{
	\pgfmathparse{#2*7+\p+0.25-#3}
        \edef\myx{\pgfmathresult}
        % draw three lines for top, right, bottom of this key
        \draw (\myx,-1) -- (\myx,-5);
        \draw (\myx,-1) -- ($(\myx,-1)+(-1,0)$);
        \draw (\myx,-5) -- ($(\myx,-5)+(-1,0)$);
        % print pitch on line
        \node [anchor=base,xshift=-15] at (\pgfmathresult,-4.5) {{#1}{#2}};
        
		\midiNote{#1}{#2}{0};
    		\node[anchor=base,xshift=-15] at (\pgfmathresult, -5.5) {\midiNumber};

        \blacknotefalse
        \ifcase\p
        \or
            \blacknotetrue
        \or
          \ifnum#2>0
            \blacknotetrue
          \fi
        \or
        \or
            \blacknotetrue
        \or
            \blacknotetrue
        \or
            \blacknotetrue
        \else
        \fi
        \ifblacknote
            % recalculate x
            \pgfmathparse{#2*7+\p-#3}
                \fill ([xshift=0.25cm, yshift=-1cm]stave.south -| \pgfmathresult,0) ++(-0.25cm,0) rectangle ++(0.5cm,-2.5cm);
                % print pitch on black key
                \node [anchor=base,xshift=0.25cm,white] at (\pgfmathresult,-2.5) {\textbf{#1}$\sharp$};
		        \midiNote{#1}{#2}{1};
    		       \node[anchor=base,xshift=0.25cm] at (\pgfmathresult, -0.75) {\midiNumber};
        \fi
}

\newif\ifblacknote
\foreach \octave in {2,...,5}
    \foreach \pitch [count=\p] in {C,D,E,F,G,A,B}{
        % calculate x position from octave and pitch
        \drawPianoKey{\pitch}{\octave}{5}{15}
    }

\def\p{7}
\drawPianoKey{C}{6}{11}{21}
\end{tikzpicture} 
