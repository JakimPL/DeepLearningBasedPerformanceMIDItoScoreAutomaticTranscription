\chapter{Overview of Existing Methods for MIDI to Score Conversion}

\section{A Brief History}

While the modern formulation of the problem appears to be present in the academic discourse since \cite{Cogliati2016}, the problem of interpreting various elements of the performance MIDI has been known much earlier \cite{Cambouropoulos2000}.

Cogliati et al. present a brief but succinct history of the development of AMT systems focused on music transcription into a score or some other intermediate forms \cite{Cogliati2016}. Some major steps involve the following papers:
\begin{itemize}
	\item Aforementioned Cambouropoulos \cite{Cambouropoulos2000} identifies musical elements and tasks needed to transcribe a performance recording into a score.
	\item Takeda et al. \cite{Takeda2002} introduce the automatic transcription of monophonic recordings using Hidden Markov Models.
	\item 
	\item In 2005, Yang \cite{Yang2005} propose a dynamic programming approach to solve beat quantization problem.
	\item Grohganz \cite{Grohganz2014} defines the concept of score-performed and score-informed MIDI files and proposes a method to transform the former to the latter.
\end{itemize}

After the formulation, the problem has been developed, incorporating the advancements from machine learning, including:
\begin{enumerate}
	\item Nakamura et al. \cite{Nakamura2017} use Markov Random Fields for piano transcription.
	\item Liu et al. \cite{Liu2022} use a deep-learning method involving convolutional neural networks and gated recurrent units, which is the main focus of the thesis.
\end{enumerate}

\section{Modern Formulation of the Problem}

While systems for performance MIDI to score systems have been existed for a while\footnote{\emph{MuseScore 3} has a special module for importing performance MIDI. \missing}, the particular implementations were not the subject of scientific publications until 2016, when Cogliati et al. defined the entire process of transcribing an unannotated and unquantized MIDI file into a score. 

Previous AMT systems focused on different aspects of the transforming a recording (either from an audio or a MIDI file) into some form of musical notation. Many of these systems recognized only so-called low level of musical information \cite{Cogliati2016}, usually focusing on only one aspect of the transcription.

\begin{figure}[!ht]
\centering
\begin{tikzpicture}
    \tikzset{font=\scriptsize}
    \tikzset{layer/.style={fill=gray!10, draw=gray, thick, text centered, align=center, minimum width=1.75cm, minimum height=1.0cm}}

    \tikzset{arrowstyle/.style={-{Latex[length=3mm]}}}
    \draw (-7.5, 0) node[layer] (midi_file) {MIDI File};
    \draw (-5.0, 0) node[layer] (note_list) {Note List};
    \draw (-2.5, 4.0) node[layer] (streams) {Streams};
    \draw (-2.5, 1.0) node[layer] (beats) {Beats};
    \draw (-2.5, 0) node[layer] (meter) {Meter};
    \draw (-2.5, -1.0) node[layer] (time_signature) {Time \\ Signature};
    \draw (-2.5, -3.0) node[layer] (harmony) {Harmony};
    \draw (-2.5, -4.0) node[layer] (key_signature) {Key \\ Signature};
    \draw (0, 0.5) node[layer] (quantized_notes) {Quantized \\ Notes};
    \draw (0, -2.0) node[layer] (note_spelling) {Note \\ Spelling};
    \draw (2.5, 4.0) node[layer] (staves) {Staves};
    \draw (2.5, 2.0) node[layer] (bars_voices) {Bars / \\ Voices};
    \draw (5.0, -1.0) node[layer] (score) {Score};

    \draw[arrowstyle] (midi_file.east) -- (note_list.west);
    \draw[arrowstyle] (note_list.north) -- (streams.south west);
    \draw[arrowstyle] (note_list.east) -- (meter.west);
    \draw[arrowstyle] (note_list.south) -- (harmony.north west);
    \draw[arrowstyle] (streams.east) -- (staves.west);
    \draw[arrowstyle] (staves.south) -- (bars_voices.north);
    \draw[arrowstyle] (beats.east) -- (bars_voices.west);
    \draw[arrowstyle] (beats.east) -- (quantized_notes.west);
    \draw[arrowstyle] (harmony.east) -- (note_spelling.south west);
    \draw[arrowstyle] (key_signature.east) -- (score.south west);
    \draw[arrowstyle] (note_spelling.east) -- (score.west);
    \draw[arrowstyle] (quantized_notes.east) -- (score.west);
    \draw[arrowstyle] (time_signature.east) -- (score.west);
    \draw[arrowstyle] (bars_voices.south east) -- (score.north west);
\end{tikzpicture}
\caption[Performance MIDI to score system proposed by Cogliati et al.]{Performance MIDI to score system proposed by Cogliati et al. \cite{Cogliati2016}.}
\end{figure}

We are going to describe some of foregoing in the previous section method.

\section{Hidden Markov Model}

\emph{Hidden Markov Model} (HMM) is a statistical framework that allows to estimate variables that are not directly observable (so called \emph{hidden} variables) but inferred through other dependent variable: \emph{observations} \cite[p. 210--213]{Jurafsky2009}.

A HMM is specified by the following components: \begin{itemize}
	\item $Q=\{q_1,q_2,\ldots, q_S\}$ is a set of (hidden) \emph{states}.
	\item $A=(a_{ij})_{i,j=1}^S$ is a \emph{transition probability matrix}, where each $a_{ij}$ represents the probability of moving from the state $i$ to the state $j$. Each row is a probability distribution, that is $\sum_{j=1}^S a_{ij}=1$ for each $i$.
	\item $X=(x_t)_{t=1}^T$ is a sequence of \emph{observations}, drawn from some random variable,
	\item $B=\{b_i(x_t)\}_{i=1}^S$ is a sequence of \emph{emission probabilities}, expressing the probability of an observation $x_t$ being generated by the state $i$.
	\item $\{\pi_i\}_{i=1}^S$ being an \emph{initial distribution over states}.
\end{itemize}

They are two underlying assumptions for the model: \begin{itemize}
	\item {\bf Markov Assumption}: The probability of a particular state depends only on the previous state: $$P\left(q_i|q_1\,\ldots,q_{i-1}\right) = P(q_i|q_{i-1})$$
	\item {\bf Output Independence}: The probability of an output observation $o_i$ depends only on the state that produced the observation $q_i$: $$P\left(x_i|q_1,\ldots,q_i,\ldots,q_T,x_1,\ldots,x_i,\ldots, x_T\right)=P(x_i|q_i)$$
\end{itemize}

Hidden Markov Models are said to be characterized by \emph{three fundamental problems} \cite[p. 213]{Jurafsky2009}: \begin{enumerate}
	\item {\bf Likelihood}: Given an HMM $\lambda = (A, B)$ and observation sequence $X$, determine the likelihood $P(X|\lambda)$
	\item {\bf Decoding}: Given an observation sequence $X$ and an HMM $\lambda=(A, B)$ determine the optimal state sequence $X$
	\item {\bf Learning}: Given an observation sequence $X$ and the set of states in the HMM, learn the model parameters $A$ and $B$.
\end{enumerate}

These problems can be solved algorithmically. For more information, refer to \cite[p. 213--226]{Jurafsky2009}.

\subsection{Hidden Markov Model for Beat Quantization}

Takeda et al. \cite{Takeda2002} employed Hidden Markov Models to solve beat quantization problem for performance MIDI streams in a tempo-free scenario. 

The basic idea behind the approach is to treat performance note durations as \emph{observations} of some unknown, ,,ideal'' (nominal, intended time value). The actual length fluctuation can be modeled by some Gaussian distribution with the mean of the length of the nominal length of a note. The performed note duration $x_t$ at the time $t$ is modeled by a probability density function $b_i(x_t)$ where $i$ is the \emph{intention}, i.e. nominal time value of the note. The observations are measured via \emph{inter-onset intervals} (IOI), that is the time interval between consecutive note onsets.

Let $Q = \{q_1,q_2,\ldots, q_N\}$ be the time sequence of identified intentions, and let $X=\{x_1,x_2,\ldots,x_N\}$ be the observations: the sequence of actual durations. The probability of observing the entire sequence is given by: $$P(X|Q)=\prod_{t=1}^N b_{q_t}(x_t)$$ 

The authors introduce two models for predicting note lengths: \begin{itemize}
	\item \emph{Note $n$-gram Model}. Note length is predicted from preceding $n - 1$ notes. For example, the bigram model predicts the length of a new note based on the length of the previous note.
	\item \emph{Rhythm Vocabulary Model}. The vocabulary consists of all known rhythm patterns for a unit time, e.g. one measure.\end{itemize}
	
Both models are represented by probabilistic state transition networks. Each state is associated with an intended note length. For example, in a $3$-gram model we may represent a state as $a_{ij,jk}$, where two preceding notes are of length $i$ and $j$, and the successor note is of length $k$. In general, instead of using tuplets, all possible stats are labeled and enumerated, yielding transition matrix $(a_{ij})_{i,j}$. Thus, the probability that state number changes along a time sequence $Q=\{q_1,q_2,\ldots,q_N\}$ is given by the product: $$P(Q)=\pi_{q_0}\prod_{t=1}^N a_{q_{t-1},q_t}$$ for some initial probability $\pi_i$ starting with the state $i$. This the foundation of Hidden Markov Model: \begin{itemize}
	\item The matrix $A=(a_{ij})_{i,j}$ constitutes a \emph{transition probability matrix}.
	\item The sequence $B=\{b_i(x_t)\}_{i}$ is a sequence of \emph{emission probabilities}.
\end{itemize}

\subsection{Optimal Sequence of States}

From the Bayes theorem, one have: $$P(Q|X)=\frac{P(X|Q)P(Q)}{P(X)}$$ In order to maximize a posteriori probability $P(Q|X)$, one needs to maximize $P(Q|X)$ as for given sequence $X$, the probability $P(X)$ is constant. One could examine all possible state sequences, compute the likelihood of the observation sequence given that hidden state sequence, and pick the one that maximizes that likelihood. However the space of all state sequences is exponentially large.

Instead, the optimal sequence of states is found through a faster \emph{Viterbi algorithm}, which is an established dynamic programming algorithm for that purpose \cite[p.210--220]{Jurafsky2009}. 

Given the HMM $\lambda = (A, B)$, the procedure is as follows:

\begin{algorithm}[ht!]
\begin{algorithmic}[1]
\Require{the HMM model, the sequence of observations $X=(x_t)_{t=1}^N$}
\Ensure{the optimal sequence of hidden states $Q=(q_1,\ldots,q_N)$}

\Statex \textbf{Initialization}
\State initialize the matrices $v$ and $bp$ of the size $S\times N$
\For{\textbf{each} state $i$ \textbf{from} $1$ \textbf{to} $S$}
	\State $v_1(i)\ \gets \pi_i\cdot b_i(x_1)$
	\State $bp_1(j)\gets 0$
\EndFor

\Statex \textbf{Recursion}
\For{\textbf{each} time step \textbf{from} $t=2$ \textbf{to} $N$}
	\For{\textbf{each} state \textbf{from} $j=1$ \textbf{to} $S$}
		\State $v_i(t) \gets \max_{j=1}^S v_{j}(t-1)\cdot a_{ji}\cdot b_i(x_t)$
		\State $bp_i(t) \gets \argmax_{j=1}^S v_j(t-1)\cdot a_{ji}\cdot b_i(x_t)$
	\EndFor
\EndFor

\Statex \textbf{Termination}
\State $p \gets \max_{j=1}^S v_j(N)$ \Comment{best path probability}
\State $q \gets \argmax_{j=1}^S v_j(N)$ \Comment{best path pointer}
\State best path $\gets$ the path starting at the state $q$, that follows the backpointer $bp$ to states back in time

    \State \Return best path, $p$
\end{algorithmic}
\captionof{algorithm}[Viterbi algorithm.]{Viterbi algorithm.}
\end{algorithm}

The procedure returns the optimal sequence of intended notes $$\hat{Q} = \argmax_Q P(X|Q)P(Q)$$

\subsection{Training}

Hidden Markov Models are subject to training in order to find the transition matrix $A$ and the sequence $B$. More precisely, given the sequence $X$ and corresponding sequence of hidden states $Q$, the model parameters $A$ and $B$ are inferred from the training data. For more information, refer to the \emph{Baum-Welch algorithm} \cite[p. 220--226]{Jurafsky2009}.

Takeda et al. used approximately $50\;000$ notes in MIDI data of classical and jazz music for the $n$-gram model, with an additional smoothing. For the rhythm vocabulary model, they extracted 267 one-bar-long rhythm patterns from 88 various pieces. The exact dataset information has been not provided.

\subsection{Limitations}

The model focuses solely on beat quantization problem, it gives no insight about other music aspects as key signature. It however, partially deals with time signature: for each key signature there is a separate model.

As the Gaussian distribution is unimodal, the basic assumption of the model is that the tempo is constant. Similarly, to catch fluctuating tempi, several different models are being run in parallel, and the one that maximizes the likelihood for the observed note durations $P(X|Q,T)P(Q|T)P(T)$ is selected, where $Q$ is the recognized rhythm, and $T$ is the tempo. Usually, a limited discrete set of predefined tempi are used\footnote{The authors used a set of 6 equally spaced tempi, ranging from 60 to 120 beats per minute.}.

The algorithm was designed mostly for transcribing of single-voice music. Authors claim that with enriched rhythm vocabulary containing overlapping notes (as in fugues or cannons) it is possible to use the framework on multi-voice music, however it needs a chord identification to treat groups of notes played at almost the same time as one entity. The paper does not explain the results of such procedure.

There is no public implementation of the algorithm available. This makes hard to compare the results with the currently available solutions.

\section{Dynamic Programming Approach}

In 2005, Yang et al. proposed another method for solving beat quantization problem \cite{Yang2005}. The authors dynamical programming approach to segment a piece into sections with different \emph{tatums}. \emph{Tatum} is often defined as ,,the smallest cognitively meaningful
subdivision of the main beat''\footnote{Usually coincides with sixteenth-, twenty-fourth- or thirty-second notes.} \cite{Iyer1997}. The resulted tatums form a grid, from which the notes are being snapped to.

\begin{figure}[!ht]
\centering
\begin{tikzpicture}
    \tikzset{layer/.style={fill=gray!10, draw=gray, thick, text centered, minimum width=5cm, minimum height=0.75cm}}
    \tikzset{arrowstyle/.style={-{Latex[length=3mm]}}}

    \draw (-3, 6.25) node[layer] (performed_audio) {Performed Audio};
    \draw (3, 6.25) node[layer] (performed_midi) {Performed MIDI};
    \draw (-3, 5.00) node[layer] (onset_detection) {Onset Detection};
    \draw (3, 5.00) node[layer] (onset_extraction) {Onset Extraction};
    \draw (0, 3.75) node[layer] (simultaneity_quantization) {Simultaneity Quantization};
    \draw (0, 2.50) node[layer] (tatum_segmentation) {Tatum Segmentation};
    \draw (0, 1.25) node[layer] (note_onset_quantization) {Note Onset Quantization};
    \draw (0, 0) node[layer] (metric_structure) {Metric Structure};

    \draw[arrowstyle] (performed_audio) -- (onset_detection);
    \draw[arrowstyle] (onset_detection) -- (simultaneity_quantization);
    \draw[arrowstyle] (performed_midi) -- (onset_extraction);
    \draw[arrowstyle] (onset_extraction) -- (simultaneity_quantization);

    \draw[arrowstyle] (simultaneity_quantization) -- (tatum_segmentation);
    \draw[arrowstyle] (tatum_segmentation) -- (note_onset_quantization);
    \draw[arrowstyle] (note_onset_quantization) -- (metric_structure);
\end{tikzpicture}
\caption[System Diagram]{System Diagram \cite{Yang2005}. The algorithm from the paper handles in fact three steps: \emph{Simultaneity Quantization}, \emph{Tatum Segmentation} and \emph{Note Onset Quantization}. Imposing the metric structure of the transcribed piece is not a goal of the algorithm.}
\end{figure}

\subsection{Simultaneity Quantization}

As the name suggests, \emph{Simultaneity Quantization} is a pre-processing step that recognizes events that are played at almost the same time, like chords. The consolidation procedure is: any group of events that lie within a certain time window is replaced by the group with a common averaged onset. 

The window needs to be carefully adjusted: too small window may not correctly recognize all simultaneous entities but too large window may collapse musical ornaments such as thrills or grace notes. For some pieces there is no viable fixed threshold that allows discerning fast arpeggiated chords from thrills: some chords that should be treated as one entity may be played in fact slower than some ornaments that needs to be separated.

\subsection{Tatum Segmentation}

The main part of the algorithm is \emph{Tatum Segmentation} for finding an optimal metric grid in a performed piece, where optimal means of the least tatum-to-note error. The algorithm takes only onsets as inputs $\mathcal{O}=\mathcal{O}_1^n=\{o_i\}_{i=1}^n$ and disregards note-off events\footnote{This means that inter-onset intervals (IOI) are being used for the beat quantization.}. The output of the algorithm is a sequence of segmentation points $S=\{S_j\}_{j=1}^m$ within optimal tatums $\{T_j\}_{j=1}^m$.

The authors define recursively the \emph{optimal segmentation function} as: $$\operatorname{OPT}(k) = \min_{1\leq i \leq k}\left[\operatorname{OPT}(i) + \operatorname{ERR}\left(\mathcal{O}_{i+1}^k\right)\right]$$ where $\mathcal{O}_{i+1}^k=\{o_{i+1},\ldots,o_k\}$ and the error function $\operatorname{ERR}$ describes the error obtained by the best tatum assignment for the sequence of onsets $\mathcal{O}_{i+1}^k$: $$\operatorname{ERR}\left(\mathcal{O}_{i+1}^k\right)=\min_p e\left(p,i+1,k\right)=\min_p \sum_{j=1}^{k-i-1}d_{\mathbb{Z}}\left(\frac{o_{j+1}-o_j}{p}\right)^2$$ where $d_{\mathbb{Z}}(x)$ is the distance of $x$ to the nearest integer, defined as $\left|x - \left\lfloor x + \tfrac{1}{2}\right\rfloor\right|$ The error function represents the remainder squared error (RSE), and the expression $o_{j+1}-o_j$ is the IOI between onsets $j$ and $j+1$. In other words, the error function describes the total error of snapping onsets to the grid. An observant reader may have noticed that the optimal tatum segmentation minimizes the RSE of the entire piece, that is $\operatorname{ERR}\left(\operatorname{\mathcal{O}}\right)$. 

Let $S(k)$ be the \emph{segmentation boundary}: an argument that minimizes the cost function: $$S(k)=\argmin_{1\leq i \leq k}\left[\operatorname{OPT}(i) + \operatorname{ERR}\left(\mathcal{O}_{i+1}^k\right)\right]$$ Similarly, let the best \emph{tatum level} minimizing $\operatorname{ERR}\left(\mathcal{O}_{i+1}^k\right)$ be: $$T(k)=\argmin_{p} e(p,i+1,k)$$ 

With the notation settled down, we can outline the main algorithm. It consists of two parts: the main procedure in which the optimal tatums are found, and the trace-back that finds the segmentation bounds. The entire algorithm is of the $O\left(n^3\right)$ time complexity.

\begin{algorithm}[ht!]
    \begin{algorithmic}[1]
        \Require{list of onsets $\mathcal{O}_1^n$}
        \Ensure{the segmentation boundaries $\{S_j\}_{j=1}^m$ and optimal tatum levels $\{T_j\}_{j=1}^m$}
        \Statex{{\bf Main procedure}}
        \State initialize tables $S[\cdot]$ and $T[\cdot]$ from $1$ to $n$
        \For{$k=1$ \textbf{to} $n$}
            \State $S[k]\gets \argmin_{1\leq i \leq k}\operatorname{OPT}(i)+\operatorname{ERR}\left(\mathcal{O}_{i+1}^k\right)$
            \State $\operatorname{OPT}(k)\gets \operatorname{OPT}(S_k)+\operatorname{ERR}\left(\mathcal{O}_{S_k+1}^k\right)$
            \State $T[S_k]\gets\argmin_p e(p,S_k+1,k)$
        \EndFor
        \Statex {\bf Trace-back}
        \State initialize sequences $S$ and $T$
        \State $k \gets n$
        \State $j \gets 0$
        \While{$k > 0$}
            \State $j \gets j + 1$
            \State $S_j \gets S[k]$
            \State $T_j \gets T[k]$
            \State $k \gets S_k$ \Comment{move to the previous segmentation point}
        \EndWhile
        \State \Return reversed $\{S_j\}_{j=1}^m$
    \end{algorithmic}
    \captionof{algorithm}[Dynamic programming tatum segmentation]{Dynamic programming tatum segmentation.}
\end{algorithm}

\subsection{Note Onset Quantization}

After the segmentation, one needs to snap the onsets to the obtained grid, for each piece segment. Additionally, the notes are snapped from left to right, and the first note indicates the first grid point. 
