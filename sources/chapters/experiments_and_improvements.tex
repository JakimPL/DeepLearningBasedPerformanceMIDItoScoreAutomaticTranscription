\chapter{Experiments and Improvements}\label{experiments_and_improvements}

In order to enhance the proposed model, we employed different strategies for finding improvements: \begin{itemize}
	\item We tried smaller and faster networks that are data efficient to optimize the current architecture.
	\item We tested more powerful architectures that proved to be successful (Transformers) in the case of natural language processing.
	\item We extended a model with an additional module for handling other musical elements: dynamics.\end{itemize}
	
Let us present results of these experiments.

\section{Transformers}

\emph{Transformers} are a class of neural network architectures introduced by Vaswani et al. in the landmark paper \emph{Attention Is All You Need} \cite{Vaswani2017}. Originally developed for natural language processing (NLP), have since become versatile and powerful framework for modeling sequential data across numerous other fields, including image processing \cite{Dosovitskiy2020}, audio generation \cite{Borsos2023} and even symbolic music analysis \cite{Zhu2021}.

\begin{figure}[ht!]
\centering
\begin{tikzpicture}
    \tikzset{block/.style={fill=gray!20, text width=3.5cm, draw=black, thick, text centered, align=center, rounded corners, minimum height=1.5em, minimum width=3.5cm}}
    \tikzset{input/.style={fill=gray!30, text width=3.5cm, draw=black, thick, text centered, rounded corners, minimum height=1.5em, minimum width=3.5cm}}
    \tikzset{operation/.style={draw, circle, thick, minimum size=0.75cm}}
    \tikzset{arrowstyle/.style={-{Latex[length=3mm]}}}

    \node (inputs) at (0, 1.5) {Inputs};
    \node (outputs) at (6, 1.5) {Shifted Outputs};

    \node[input] (input_embed) at (0, 0) {Input Embedding};
    \node[block] (pos_encoding) at (-3, -1.2) {Positional \\ Encoding};
    \node[operation] (add1) at (0, -1.2) {$+$};

    \node[block] (multihead1) at (0, -3.25) {Multi-Head \\ Attention};
    \node[block] (add_norm1) at (0, -4.5) {Add \& Norm};
    \node[block] (ffn1) at (0, -6.25) {Feed \\ Forward};
    \node[block] (add_norm2) at (0, -7.5) {Add \& Norm};

    \draw[rounded corners, thick] (-2.75, -2) rectangle (2.75, -8.25) node[pos=0.5, xshift=-3.1cm] {$N \times \;$};

    \node[input] (output_embed) at (6, 0) {Output Embedding};
    \node[block] (pos_encoding_dec) at (9, -1.2) {Positional \\ Encoding};
    \node[operation] (add2) at (6, -1.2) {$+$};

    \node[block] (masked_mha) at (6, -3.25) {Masked Multi-Head \\ Attention};
    \node[block] (add_norm_dec1) at (6, -4.5) {Add \& Norm};
    \node[block] (mha_dec) at (6, -6.40) {Multi-Head \\ Attention};
    \node[block] (add_norm_dec2) at (6, -7.50) {Add \& Norm};
    \node[block] (ffn2) at (6, -9.25) {Feed \\ Forward};
    \node[block] (add_norm_dec3) at (6, -10.50) {Add \& Norm};

    \draw[rounded corners, thick] (3.25, -2) rectangle (8.75, -11.25) node[pos=0.5, xshift=3.1cm] {$\, \times N$};

    \node[block] (linear) at (6, -12.2) {Linear};
    \node[block] (softmax) at (6, -13.6) {Softmax};
    \node (output_probs) at (6, -15.1) {Output Probabilities};

    \draw[arrowstyle] (inputs.south) -- (input_embed.north);
    \draw[arrowstyle] (input_embed.south) -- (add1.north);
    \draw[arrowstyle] (pos_encoding.east) -- (add1.west);
    \draw[arrowstyle] (add1) -- (multihead1);
    \draw[arrowstyle] (multihead1) -- (add_norm1);
    \draw[arrowstyle] (add_norm1) -- (ffn1);
    \draw[arrowstyle] (ffn1) -- (add_norm2);

    \draw[arrowstyle] (outputs.south) -- (output_embed.north);
    \draw[arrowstyle] (output_embed.south) -- (add2.north);
    \draw[arrowstyle] (pos_encoding_dec.west) -- (add2.east);
    \draw[arrowstyle] (add2) -- (masked_mha);
    \draw[arrowstyle] (masked_mha) -- (add_norm_dec1);
    \draw[arrowstyle] (mha_dec) -- (add_norm_dec2);
    \draw[arrowstyle] (add_norm_dec2) -- (ffn2);
    \draw[arrowstyle] (ffn2) -- (add_norm_dec3);
    \draw[arrowstyle] (add_norm_dec3) -- (linear);
    \draw[arrowstyle] (linear) -- (softmax);
    \draw[arrowstyle] (softmax) -- (output_probs);

    \draw[arrowstyle] (add_norm2.east) -| ++(0.325, 0) |- ++(0, 2.05) -| ($(mha_dec.north) - (1.0, 0.0)$);
    \draw[arrowstyle] (add_norm2.east) -| ++(0.325, 0) |- ++(0, 2.05) -| (mha_dec.north);

    \draw[arrowstyle] (0, -2.25) -| ++(-2.325,0) |- (add_norm1.west);
    \draw[arrowstyle] (0, -5.25) -| ++(-2.325,0) |- (add_norm2.west);

    \draw[arrowstyle] (0, -2.35) -| ++(-1.0,0) -- ($(multihead1.north) - (1.0, 0.0)$);
    \draw[arrowstyle] (0, -2.35) -| ++(1.0,0) -- ($(multihead1.north) + (1.0, 0.0)$);

    \draw[arrowstyle] (6, -2.35) -| ++(-1.0,0) -- ($(masked_mha.north) - (1.0, 0.0)$);
    \draw[arrowstyle] (6, -2.35) -| ++(1.0,0) -- ($(masked_mha.north) + (1.0, 0.0)$);

    \draw[arrowstyle] (add_norm_dec1) -- (6, -5.25) -| ++(1.0,0) -- ($(mha_dec.north) + (1.0, 0.0)$);

    \draw[arrowstyle] (6, -2.25) -| ++(2.325,0) |- (add_norm_dec1.east);
    \draw[arrowstyle] (6, -5.25) -| ++(2.325,0) |- (add_norm_dec2.east);
    \draw[arrowstyle] (6, -8.25) -| ++(2.325,0) |- (add_norm_dec3.east);

\end{tikzpicture}
\caption[The Transformer architecture.]{The Transformer architecture \cite{Vaswani2017}.}
\end{figure}

The key innovation in Transformers is the \emph{self-attention} mechanism, which enables the model to weigh the relevance of each part of an input sequence relative to other parts, regardless of their distance from each other in the sequence. Unlike recurrent neural networks, which process data sequentially and can struggle with long-range dependencies, Transformers use that mechanism to directly model these dependencies in parallel.

As the authors state at the end of the paper \cite{Liu2022}: \begin{quote}Possible next steps include investigating more powerful model architectures such as the Transformer.\end{quote} We analyzed various setups involving Transformer and attention mechanisms, preserving the training and evaluation setup.

\subsection{Vanilla Transformer}

\missing

% We hypothesize that Transformer architecture could outperform the proposed network but in a data-abundant scenario, but the entire dataset is still too small and needs more data-efficient models.

% Moreover, the models' assignment heavy rely on small context of the piece. In other words, musical information that lies far apart from notes is not particularly useful for the inference.

\subsection{Attention}

\missing

% We hypothesize that Transformer architecture could outperform the proposed network but in a data-abundant scenario, but the entire dataset is still too small and needs more data-efficient models.

% Moreover, the models' assignment heavy rely on small context of the piece. In other words, musical information that lies far apart from notes is not particularly useful for the inference.

\section{Temporal Convolutional Network}

\missing

\begin{figure}[ht!]
\centering
\begin{tikzpicture}
    \tikzset{layer1/.style={fill=gray!10, draw=gray, thick, text centered}}
    \tikzset{layer2/.style={fill=gray!30, draw=gray, thick, text centered}}
    \tikzset{operation/.style={draw, circle, thick, minimum size=0.75cm}}
    \tikzset{arrowstyle/.style={-{Latex[length=3mm]}}}

    \fill[fill=gray!50, fill opacity=0.5, draw=gray, thick, line width=2pt] (0.0, -2.0) rectangle (12.0, 7.0);
    \fill[fill=white!50, fill opacity=1.0, draw=black, thick, line width=1pt] (0.5, 0.0) rectangle (10.5, 6.5);
    \fill[fill=white!50, fill opacity=1.0, draw=black, thick, line width=1pt] (1.0, -0.5) rectangle (11.0, 6.0);
    \fill[fill=white!50, fill opacity=1.0, draw=black, thick, line width=1pt] (1.5, -1.0) rectangle (11.5, 5.5);

    \draw (6.5, 9.00) node (note_sequence) {Note sequence};
    \draw (6.5, 8.00) node[layer1] (embedding) {Linear};
    \draw (4.0, 4.50) node[layer1] (layerA1) {Dilated Convolutional};
    \draw (4.0, 3.50) node[layer1] (layerA2) {ReLU};
    \draw (4.0, 2.50) node[layer1] (layerA3) {Dropout};
    \draw (4.0, 1.50) node[layer1] (layerA4) {Dilated Convolutional};
    \draw (6.5, 0.75) node[operation] (add) {$+$};
    \draw (6.5, -0.50) node[layer2] (layerA5) {Layer Normalization};
    \draw[anchor=south, inner sep=2pt] (6.5, -2.0) node (outputA) {To the next convolutional layer};

    \draw (6.5, -3.00) node[layer1] (output) {Linear};

    \node[anchor=south west, inner sep=2pt] at (0.2, -1.8) {TCNBlock};

    \node[anchor=north west, inner sep=2pt] at (1.55, 5.45) {\small{9, 1, 256}};
    \node[anchor=north west, inner sep=2pt] at (1.05, 5.95) {\small{9, 2, 256}};
    \node[anchor=north west, inner sep=2pt] at (0.55, 6.45) {\small{9, 4, 256}};
    \node[anchor=north west, inner sep=2pt] at (0.05, 6.95) {\small{kernel size, dilation, hidden size}};

    \draw[arrowstyle] (note_sequence.south) to (embedding.north);
    \draw[arrowstyle] (embedding.south) -| ++(0.0, -2.5) -| (layerA1.north);
    \draw[arrowstyle] (embedding.south) -| ++(0.0, -2.5) -| ++(2.5, 0.0) |- (add.east);
    \draw[arrowstyle] (layerA1.south) to (layerA2.north);
    \draw[arrowstyle] (layerA2.south) to (layerA3.north);
    \draw[arrowstyle] (layerA3.south) to (layerA4.north);
    \draw[arrowstyle] (layerA4.south) |- (add.west);
    \draw[arrowstyle] (add.south) to (layerA5.north);
    \draw[arrowstyle] (layerA5.south) to (outputA.north);
    \draw[arrowstyle] (outputA.south) to (output.north);
\end{tikzpicture}

\caption[The Temporal Convolutional Block.]{The Temporal Convolutional Block.}
\end{figure}

\section{Dynamics}

\missing

\begin{figure}[!ht]
\centering
\begin{tikzpicture}
    \tikzset{sequence/.style={thick, text centered, align=center}}
    \tikzset{convblock/.style={fill=gray!20, draw=gray, thick, text centered}}
    \tikzset{grublock/.style={fill=gray!30, draw=gray, thick, text centered}}
    \tikzset{linear/.style={fill=gray!10, draw=gray, thick, text centered}}
    \tikzset{output/.style={thick, anchor=west, align=left}}
    \tikzset{arrowstyle/.style={-{Latex[length=3mm]}}}
    \draw (-3.5, 0.5) node[sequence] (sequence) {note\\ sequence};

    \draw (-1.75, 5) node (joint1) {};
    \draw (-1.75, 3) node (joint2) {};
    \draw (-1.75, 2) node (joint3) {};
    \draw (-1.75, 0.5) node (joint4) {};
    \draw (-1.75, -1) node (joint5) {};
    \draw (-1.75, -2) node (joint6) {};
    \draw (-1.75, -3) node (joint7) {};

    \draw (1.5, 3) node (concatenation) {$\boldsymbol{\oplus}$};
    \draw (1.5, 3.5) node (concatleftjoint1) {};
    \draw (1.5, 2.5) node (concatleftjoint2) {};
    \draw (7.25, 3.5) node (concatrightjoint1) {};
    \draw (7.25, 2.5) node (concatrightjoint2) {};
    \draw (7.25, 2) node (concatstart1) {};
    \draw (7.25, 4) node (concatstart2) {};

    \draw (0, 5) node[convblock] (convblock1) {ConvBlock};
    \draw (0, 3) node[convblock] (convblock2) {ConvBlock};
    \draw (0, 2) node[convblock] (convblock3) {ConvBlock};
    \draw (0, 0.5) node[convblock] (convblock4) {ConvBlock};
    \draw (0, -1) node[convblock] (convblock5) {ConvBlock};
    \draw (0, -2) node[convblock] (convblock6) {ConvBlock};
    \draw (0, -3) node[convblock] (convblock7) {ConvBlock};

    \draw (3, 6) node[grublock] (grublock1) {GRUBlock};
    \draw (3, 5) node[grublock] (grublock2) {GRUBlock};
    \draw (3, 4) node[grublock] (grublock3) {GRUBlock};
    \draw (3, 3) node[grublock] (grublock4) {GRUBlock};
    \draw (3, 2) node[grublock] (grublock5) {GRUBlock};
    \draw (3, 0.5) node[grublock] (grublock6) {GRUBlock};
    \draw (3, -1) node[grublock] (grublock7) {GRUBlock};
    \draw (3, -2) node[grublock] (grublock8) {GRUBlock};
    \draw (3, -3) node[grublock] (grublock9) {GRUBlock};

    \draw (6, 6) node[linear] (linear1) {Linear (200)};
    \draw (6, 5) node[linear] (linear2) {Linear (1)};
    \draw (6, 4) node[linear] (linear3) {Linear (1)};
    \draw (6, 3) node[linear] (linear4) {Linear (24)};
    \draw (6, 2) node[linear] (linear5) {Linear (96)};
    \draw (6, 1) node[linear] (linear6) {Linear (5)};
    \draw (6, 0) node[linear] (linear7) {Linear (4)};
    \draw (6, -1) node[linear] (linear8) {Linear (12)};
    \draw (6, -2) node[linear] (linear9) {Linear (1)};
    \draw (6, -3) node[linear] (linear10) {Linear (10)};

    \draw (7.75, 6) node[output] (output1) {tempo};
    \draw (7.75, 5) node[output] (output2) {downbeats};
    \draw (7.75, 4) node[output] (output3) {beats};
    \draw (7.75, 3) node[output] (output4) {musical onsets};
    \draw (7.75, 2) node[output] (output5) {note values};
    \draw (7.75, 1) node[output] (output6) {time signature\\  numerators};
    \draw (7.75, 0) node[output] (output7) {time signature\\  denominators};
    \draw (7.75, -1) node[output] (output8) {key signature};
    \draw (7.75, -2) node[output] (output9) {hand parts};
    \draw (7.75, -3) node[output] (output10) {dynamics};

    \draw (sequence) to (joint4.center);
    \draw (joint1.center) to (joint7.center);
    \draw[arrowstyle] (joint1.center) to (convblock1);
    \draw[arrowstyle] (joint2.center) to (convblock2);
    \draw[arrowstyle] (joint3.center) to (convblock3);
    \draw[arrowstyle] (joint4.center) to (convblock4);
    \draw[arrowstyle] (joint5.center) to (convblock5);
    \draw[arrowstyle] (joint6.center) to (convblock6);
    \draw[arrowstyle] (joint7.center) to (convblock7);

    \draw[arrowstyle] (convblock1.east) to (grublock1.west);
    \draw[arrowstyle] (convblock1.east) to (grublock2.west);
    \draw[arrowstyle] (convblock1.east) to (grublock3.west);
    \draw[arrowstyle] (convblock2.east) to (grublock4.west);
    \draw[arrowstyle] (convblock3.east) to (grublock5.west);
    \draw[arrowstyle] (convblock4.east) to (grublock6.west);
    \draw[arrowstyle] (convblock5.east) to (grublock7.west);
    \draw[arrowstyle] (convblock6.east) to (grublock8.west);
    \draw[arrowstyle] (convblock7.east) to (grublock9.west);

    \draw[arrowstyle] (grublock1.east) to (linear1.west);
    \draw[arrowstyle] (grublock2.east) to (linear2.west);
    \draw[arrowstyle] (grublock3.east) to (linear3.west);
    \draw[arrowstyle] (grublock4.east) to (linear4.west);
    \draw[arrowstyle] (grublock5.east) to (linear5.west);
    \draw[arrowstyle] (grublock6.east) to (linear6.west);
    \draw[arrowstyle] (grublock6.east) to (linear7.west);
    \draw[arrowstyle] (grublock7.east) to (linear8.west);
    \draw[arrowstyle] (grublock8.east) to (linear9.west);
    \draw[arrowstyle] (grublock9.east) to (linear10.west);

    \draw[arrowstyle] (linear1.east) to (output1.west);
    \draw[arrowstyle] (linear2.east) to (output2.west);
    \draw[arrowstyle] (linear3.east) to (output3.west);
    \draw[arrowstyle] (linear4.east) to (output4.west);
    \draw[arrowstyle] (linear5.east) to (output5.west);
    \draw[arrowstyle] (linear6.east) to (output6.west);
    \draw[arrowstyle] (linear7.east) to (output7.west);
    \draw[arrowstyle] (linear8.east) to (output8.west);
    \draw[arrowstyle] (linear9.east) to (output9.west);
    \draw[arrowstyle] (linear10.east) to (output10.west);

    \draw[arrowstyle, dashed] (concatleftjoint1.center) to (concatenation.center);
    \draw[arrowstyle] (concatleftjoint2.center) to (concatenation.center);
    \draw[dashed] (concatleftjoint1.center) to (concatrightjoint1.center);
    \draw (concatleftjoint2.center) to (concatrightjoint2.center);
    \draw (concatstart1.center) to (concatrightjoint2.center);
    \draw[dashed] (concatstart2.center) to (concatrightjoint1.center);

    \draw (3.5, -4) node[output] (legendsymbol1) {$\boldsymbol{\oplus}$};
    \draw (3, -4.75) node[output] (legendsymbolstart) {};
    \draw (4.5, -4.75) node[output] (legendsymbolend) {};
    \draw (5, -4) node[output] (legend1) {concatenation};
    \draw (5, -4.75) node[output] (legend2) {disable backpropagation during training};
    \draw[arrowstyle, dashed] (legendsymbolstart.center) to (legendsymbolend.center);
\end{tikzpicture}

\caption[The extended architecture of the model.]{The extended architecture of the model by the \emph{dynamics} model.}
\end{figure}