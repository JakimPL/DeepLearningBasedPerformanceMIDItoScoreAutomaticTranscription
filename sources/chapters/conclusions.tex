\chapter{Conclusions}

We presented the performance MIDI to score automatic transcription task in detail, described the components and challenges in that problem, presented earlier methods. Finally, we presented the state-of-the-art deep-learning model with in-depth analysis of its performance and behavior.

The analysis consisted of: \begin{itemize}
	\item Analyzing the performance and behavior of the main considered model.
	\item Providing ablation studies due to found model anomalies, especially concerning undesired influence of velocity feature.
	\item Developing local feature importance methods for further 
	\item Comparing the results with other model architectures such as Transformers or Temporal Convolutional Networks.
\end{itemize}

Additionally, we provided an enhancement to the model that allows to interpret an additional musical feature --- dynamics. 

Let us briefly summarize the results.

\section{Robustness Analysis}

We were able to discover certain artifacts of the score generation models, especially when it comes to velocity contribution to hand part assignment. The time signature model is not fully robust to alterations that should not have affect the output. Only the key signature model was robust to all considered transformation.

We also were able to mitigate the undesired behaviors without sacrificing the model's performance by data augmentation or feature removal.

\section{Feature Importance Methods}

Current XAI methods do not work well with symbolic music data in general. Developing more tailored and adaptable XAI methods for musical applications could contribute to improved model interpretability. 

We provided two ways of analyzing the behavior of submodels: semi-LIME approach that allowed us to see how velocity feature influenced the hand part assignment of other notes.

The second method, based on note omission, showed how certain notes voted for or against the current key signature assignment. We were able to verify the correctness of model decision 

Unfortunately, the proposed methods have drawbacks and are not fully justified. One of the challenges would be to find a reasonable (and interpretable) embedding of the pitch space that encodes musical features. 

\section{Experiments}

We conducted a series of experiments in which compared the state-of-the-art performance with some other solutions. There were three most promising architectures: \begin{itemize}
	\item Transformer.
	\item Self-attention.
	\item Temporal Convolutional Network.
\end{itemize}

We are going to briefly describe all these experiments.

\subsection{Transformers}

Vanilla Transfomers didn't perform as good as other networks. We hypothesize that Transformer architecture could outperform the proposed network but in a data-abundant scenario. In the considered scenario the entire dataset is though too small (or not diverse enough) and needs more data-efficient models. The current dataset is rather small, and there is a need for more high-quality annotated performance MIDI data that is diverse and representative enough. For a limited amount of data, Transformers may be prone to overfitting.

Another suspected reason is that some features like hand part assignment are local, and long-term dependencies are not that useful as they could be. 

However, recently there were a successful use of Transformers in the field \cite{Beyer2024} on a subset of the considered dataset. The authors provide a Roformer encoder-decoder model with a custom token embedding. They also provided additional data augmentations as duration and onset jitter, introducing noise to the note beginning and ends. The authors modified the initial ACPAS dataset as they claim that a hand-crafted part of the is not representative for the performance. 

This suggests that the traditional trigonometric positional encoding does not serve the purpose well, and some other form of positional encodings may significantly improve the quality of the model.

However, the aforementioned work operates on different score format (MusicXML) and uses a different set of metrics than \cite{Liu2022}, making hard to compare both models without further investigation.

\subsection{Dynamics}

% We enhanced the base model by the dynamics model which gives a 

% However, as the datasets usually don't provide the information on 