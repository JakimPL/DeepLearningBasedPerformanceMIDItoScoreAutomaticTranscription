\chapter{Performance MIDI-to-Score Conversion by Neural Beat Tracking}

The main model of interest \cite{Liu2022} is composed of two parts:
\begin{itemize}
	\item Beat Tracking.
	\item Score Element Assignment.
\end{itemize}

The first part is responsible for quantizing the raw material into beats and downbeats. Each measure is consisted of several beats, and the number of beats in a measure is governed by the time signature. The beats mark the tempo, and they are the basis for the musical onsets detection.

The second part assigns every other aspect of the musical score to the MIDI stream. Let us recall these elements: musical onsets, note values, hand parts, key signature and time signature.

Let us review both components.

\section{Beat Tracking}

A single measure consists of several beats, and the rhythmic structure of beats is determined by time signature (or signatures). A performance MIDI does not contain information about beats and one of the objectives of the model is to predict locations of these.

After these predictions, note beginnings (onsets) can be placed in a subdivision of a single beat. The smallest unit of time distance is thus dictated by the size of the beat subdivision. More precisely, a musical onset $mo_i$ of the $i$-th note is defined as \[mo_i = \frac{s_n}{S}\] where $S$ is the number of subdivisions per beat in rhythm quantization, and $s_n$ is the musical onset time expressed in the number of subdivisions.

The authors provide a novel approach to detecting beats, splitting them into two categories, for which there are two separate methods.

Beats can fall into two distinct groups:
\begin{itemize}
	\item \emph{In-note}: beats concurrent with at least one note onset. This means that some notes play within the start of the beat.
	\item \emph{Out-of-note}: a beat is not concurrent with any note onset. No notes are beginning with such beats.
\end{itemize}

\subsection{In-note Beat Prediction}

For predicting in-note beats, one can reduce the problem to a binary classification task on the note sequence $\mathbf{X}$. This sequence is assumed to have a length $N$.

\begin{figure}[!ht]
\centering
\begin{tikzpicture}
\tikzset{layer1/.style={fill=gray!10, draw=gray, thick, text centered}}
\tikzset{layer2/.style={fill=gray!30, draw=gray, thick, text centered}}
\tikzset{arrowstyle/.style={-{Latex[length=3mm]}}}

	\fill[fill=gray!50, fill opacity=0.5, draw=gray, thick, line width=2pt] (0.0, 0.0) rectangle (12.0, 7.0);
	\fill[fill=gray!30, fill opacity=0.5, draw=gray, thick, line width=2pt] (0.0, -0.5) rectangle (12.0, -4.0);
	\fill[fill=white!50, fill opacity=1.0, draw=black, thick, line width=1pt] (0.5, 2.0) rectangle (10.5, 6.5);
	\fill[fill=white!50, fill opacity=1.0, draw=black, thick, line width=1pt] (1.0, 1.5) rectangle (11.0, 6.0);
	\fill[fill=white!50, fill opacity=1.0, draw=black, thick, line width=1pt] (1.5, 1.0) rectangle (11.5, 5.5);

    \draw (6.5, 7.75) node (note_sequence) {Note sequence};
    \draw (6.5, 4.75) node[layer2] (layerA1) {Convolutional};
    \draw (6.5, 3.75) node[layer1] (layerA2) {Batch normalization};
    \draw (6.5, 2.75) node[layer1] (layerA3) {Exponential linear unit};
    \draw (6.5, 1.75) node[layer1] (layerA4) {Dropout};
    \draw[anchor=south, inner sep=2pt] (6.5, 0.0) node (outputA) {To the next convolutional layer};
    
    \draw (6.5, -1.25) node[layer2] (layerB1) {Bi-directional GRU (512)};
    \draw (6.5, -2.25) node[layer2] (layerB2) {Bi-directional GRU (512)};
    \draw (6.5, -3.25) node[layer1] (layerB3) {Linear (512)};
    
    \draw (6.5, -5.00) node[layer1] (output) {Linear (1)};
    
	\node[anchor=south west, inner sep=2pt] at (0.2, 0.2) {ConvBlock};    
	\node[anchor=south west, inner sep=2pt] at (0.2, -3.8) {GRUBlock};
	
	\node[anchor=north west, inner sep=2pt] at (1.55, 5.45) {\small{(9, in features), 128}};
	\node[anchor=north west, inner sep=2pt] at (1.05, 5.95) {\small{(9, 1), 256}};
	\node[anchor=north west, inner sep=2pt] at (0.55, 6.45) {\small{(9, 1), 512}};
	\node[anchor=north west, inner sep=2pt] at (0.05, 6.95) {\small{kernel size, channels}};
	
	\draw[arrowstyle] (note_sequence.south) to (layerA1.north);
	\draw[arrowstyle] (layerA1.south) to (layerA2.north);
	\draw[arrowstyle] (layerA2.south) to (layerA3.north);
	\draw[arrowstyle] (layerA3.south) to (layerA4.north);
	\draw[arrowstyle] (layerA4.south) to (outputA.north);
	\draw[arrowstyle] (layerA4.south) to (outputA.north);
	
	\draw[arrowstyle] (outputA.south) to (layerB1.north);
	\draw[arrowstyle] (layerB1.south) to (layerB2.north);
	\draw[arrowstyle] (layerB2.south) to (layerB3.north);
	\draw[arrowstyle] (layerB3.south) to (output.north);
\end{tikzpicture}
\caption[In-note beat prediction model]{In-note beat prediction model with a CRNN with 3 convolutional layers and 2 bi-directional gated recurrent unit (GRU) layers.}
\end{figure}

The probability $P_n$ that $n$-th note is concurrent with a beat is defined as: $$P_n = \mathbb{P}\left(B_n|\mathbf{X}\right)$$ where $B_n\in\{0,1\}$ is the ground-truth beat label for the $n$-th note from the note sequence. The model is trained using the standard cross-entropy loss function: $$\mathcal{L}=-\frac{1}{N}\sum_{n=1}^N B_n\ln P_n + \left(1-B_n\right)\ln\left(1-P_n\right)$$ The authors used CRNN with 3 convolutional layers and 2 bidirectional gated recurrent unit (GRU) layers. The probability threshold for positive classification has been set dynamically, depending on the maximum probability in a fixed segment length.

\subsection{Out-of-note Beat Prediction}

The approach to predicting out-of-note beats, which do not align with note onsets, requires distinct approach. Liu et al. (2022) proposed a dynamic programming strategy to solve this problem \cite{Liu2022}.

Let us assume that there are $B^i$ in-note beats $\{b_n^i\}_{n=1}^{B^i}$ in total, and out-of-note beats are at subdivisions of the neighboring in-note beats\footnote{We may select only one note per in-note beat, if there are more.} $b_{n}^i$ and $b_{n+1}^i$.

The goal of the procedure is to find out-of-note beats $b^o$ from a set of candidates: \begin{equation}\label{out_of_note_candidates}
b_{n,K}^o = \left\{b_n^i + \frac{k}{K+1}\left(b_{n+1}^i-b_n^i\right)\right\}_{k=1}^K
\end{equation} where $K\in\{0,1,2,3\}$ is the number of out-of-note beats to insert inside the $\left(b_{n+1}^i, b_n^i\right)$ interval.

The number of candidates is selected in order to minimize the tempo change after adding out-of-note beats. The function may be represented as the sum: \[\mathcal{O}_1 = \sum_{n=1}^{B-2}\left|\ln\left(\frac{b_{n+2} - b_{n+1}}{b_{n+1} - b_n}\right)\right|\] where $\{b_n\}_{n=1}^B$ is the sequence of all $B$ beats (both in-note and out-of-note, sorted chronologically).

To discourage the procedure from adding too many out-of-note beats, which leads to an unnecessarily subdivided output, an additional penalty is associated with the objective function: \begin{equation}\label{out_of_note_objective}
\mathcal{O} = \mathcal{O}_1 + \lambda B^o
\end{equation} where $B^o$ is the number of added out-of-note beats, and $\lambda$ is the penalty coefficient. The coefficient is to be found experimentally, however the default value set by the authors is $1$.

\begin{algorithm}[ht!]
\begin{algorithmic}[1]
\Require{list of in-note notes $\mathcal{B}^i$}
\Ensure{list of all beats $\mathcal{B}$ including added out-of-note beats}
\State $n \gets 1$
\For{$K=0,1,2,3$}
    \State initialize objective function $\mathcal{O}_K\gets 0$
    \State initialize beat sequence $\mathcal{B}_K\gets \{b_1\}$
\EndFor
\For{$n=1,2,\ldots,B^i-2$}
    \For{$K_{\textrm{cur}}=0,1,2,3$}
        \State get out-of-note beats for current step by \eqref{out_of_note_candidates}
        \If{tempo is beyond tempo range limits}
            \State \textbf{continue}
        \EndIf
        \For{$K_{\textrm{prev}}=0,1,2,3$}
            \State update objective function by \eqref{out_of_note_objective}
        \EndFor
        \State select the minimum objective among all $K_{\textrm{prev}}$ values
        \State add out-of-beats for the current step to the beat sequence mapped to the selected $K_{\textrm{prev}}$
    \EndFor
    \For{$K_{\textrm{cur}}=0,1,2,3$}
        \State update $\mathcal{O}_K$, $\mathcal{B}_K$ mapped with $K_{\textrm{cur}}$
    \EndFor
\EndFor
\State \Return{the beat sequence $\mathcal{B}_K$ with the minimum objective function $\mathcal{O}_K$}
\end{algorithmic}
\captionof{algorithm}[Out-of-note beat prediction.]{Out-of-note beat prediction.}
\end{algorithm}


\section{Score Elements Assignment}

As discussed in the Section \ref{music_score_encoding}, score elements assignment may be viewed as a function from a note sequence $\mathbf{X}$ into a score encoded by a tensor $\mathbf{Y}_n$ representing musical onsets, note values, key signature, time signature, and hand parts, for each note separately.

Besides the quantization model, which relies on the beat tracking module, key signature, time signature and hand parts modules can be treated independently.

\begin{figure}[!ht]
\centering
\begin{tikzpicture}
    \tikzset{sequence/.style={thick, text centered, align=center}}
    \tikzset{convblock/.style={fill=gray!20, draw=gray, thick, text centered}}
    \tikzset{grublock/.style={fill=gray!30, draw=gray, thick, text centered}}
    \tikzset{linear/.style={fill=gray!10, draw=gray, thick, text centered}}
    \tikzset{output/.style={thick, anchor=west, align=left}}
    \tikzset{arrowstyle/.style={-{Latex[length=3mm]}}}
    \draw (-3.5, 0.5) node[sequence] (sequence) {note\\ sequence};

    \draw (-1.75, 5) node (joint1) {};
    \draw (-1.75, 3) node (joint2) {};
    \draw (-1.75, 2) node (joint3) {};
    \draw (-1.75, 0.5) node (joint4) {};
    \draw (-1.75, -1) node (joint5) {};
    \draw (-1.75, -2) node (joint6) {};

    \draw (1.5, 3) node (concatenation) {$\boldsymbol{\oplus}$};
    \draw (1.5, 3.5) node (concatleftjoint1) {};
    \draw (1.5, 2.5) node (concatleftjoint2) {};
    \draw (7.25, 3.5) node (concatrightjoint1) {};
    \draw (7.25, 2.5) node (concatrightjoint2) {};
    \draw (7.25, 2) node (concatstart1) {};
    \draw (7.25, 4) node (concatstart2) {};

    \draw (0, 5) node[convblock] (convblock1) {ConvBlock};
    \draw (0, 3) node[convblock] (convblock2) {ConvBlock};
    \draw (0, 2) node[convblock] (convblock3) {ConvBlock};
    \draw (0, 0.5) node[convblock] (convblock4) {ConvBlock};
    \draw (0, -1) node[convblock] (convblock5) {ConvBlock};
    \draw (0, -2) node[convblock] (convblock6) {ConvBlock};

    \draw (3, 6) node[grublock] (grublock1) {GRUBlock};
    \draw (3, 5) node[grublock] (grublock2) {GRUBlock};
    \draw (3, 4) node[grublock] (grublock3) {GRUBlock};
    \draw (3, 3) node[grublock] (grublock4) {GRUBlock};
    \draw (3, 2) node[grublock] (grublock5) {GRUBlock};
    \draw (3, 0.5) node[grublock] (grublock6) {GRUBlock};
    \draw (3, -1) node[grublock] (grublock7) {GRUBlock};
    \draw (3, -2) node[grublock] (grublock8) {GRUBlock};

    \draw (6, 6) node[linear] (linear1) {Linear (200)};
    \draw (6, 5) node[linear] (linear2) {Linear (1)};
    \draw (6, 4) node[linear] (linear3) {Linear (1)};
    \draw (6, 3) node[linear] (linear4) {Linear (24)};
    \draw (6, 2) node[linear] (linear5) {Linear (96)};
    \draw (6, 1) node[linear] (linear6) {Linear (5)};
    \draw (6, 0) node[linear] (linear7) {Linear (4)};
    \draw (6, -1) node[linear] (linear8) {Linear (12)};
    \draw (6, -2) node[linear] (linear9) {Linear (1)};

    \draw (7.75, 6) node[output] (output1) {tempo};
    \draw (7.75, 5) node[output] (output2) {downbeats};
    \draw (7.75, 4) node[output] (output3) {beats};
    \draw (7.75, 3) node[output] (output4) {musical onsets};
    \draw (7.75, 2) node[output] (output5) {note values};
    \draw (7.75, 1) node[output] (output6) {time signature\\  numerators};
    \draw (7.75, 0) node[output] (output7) {time signature\\  denominators};
    \draw (7.75, -1) node[output] (output8) {key signature};
    \draw (7.75, -2) node[output] (output9) {hand parts};

    \draw (sequence) to (joint4.center);
    \draw (joint1.center) to (joint6.center);
    \draw[arrowstyle] (joint1.center) to (convblock1);
    \draw[arrowstyle] (joint2.center) to (convblock2);
    \draw[arrowstyle] (joint3.center) to (convblock3);
    \draw[arrowstyle] (joint4.center) to (convblock4);
    \draw[arrowstyle] (joint5.center) to (convblock5);
    \draw[arrowstyle] (joint6.center) to (convblock6);

    \draw[arrowstyle] (convblock1.east) to (grublock1.west);
    \draw[arrowstyle] (convblock1.east) to (grublock2.west);
    \draw[arrowstyle] (convblock1.east) to (grublock3.west);
    \draw[arrowstyle] (convblock2.east) to (grublock4.west);
    \draw[arrowstyle] (convblock3.east) to (grublock5.west);
    \draw[arrowstyle] (convblock4.east) to (grublock6.west);
    \draw[arrowstyle] (convblock5.east) to (grublock7.west);
    \draw[arrowstyle] (convblock6.east) to (grublock8.west);

    \draw[arrowstyle] (grublock1.east) to (linear1.west);
    \draw[arrowstyle] (grublock2.east) to (linear2.west);
    \draw[arrowstyle] (grublock3.east) to (linear3.west);
    \draw[arrowstyle] (grublock4.east) to (linear4.west);
    \draw[arrowstyle] (grublock5.east) to (linear5.west);
    \draw[arrowstyle] (grublock6.east) to (linear6.west);
    \draw[arrowstyle] (grublock6.east) to (linear7.west);
    \draw[arrowstyle] (grublock7.east) to (linear8.west);
    \draw[arrowstyle] (grublock8.east) to (linear9.west);

    \draw[arrowstyle] (linear1.east) to (output1.west);
    \draw[arrowstyle] (linear2.east) to (output2.west);
    \draw[arrowstyle] (linear3.east) to (output3.west);
    \draw[arrowstyle] (linear4.east) to (output4.west);
    \draw[arrowstyle] (linear5.east) to (output5.west);
    \draw[arrowstyle] (linear6.east) to (output6.west);
    \draw[arrowstyle] (linear7.east) to (output7.west);
    \draw[arrowstyle] (linear8.east) to (output8.west);
    \draw[arrowstyle] (linear9.east) to (output9.west);

    \draw[arrowstyle, dashed] (concatleftjoint1.center) to (concatenation.center);
    \draw[arrowstyle] (concatleftjoint2.center) to (concatenation.center);
    \draw[dashed] (concatleftjoint1.center) to (concatrightjoint1.center);
    \draw (concatleftjoint2.center) to (concatrightjoint2.center);
    \draw (concatstart1.center) to (concatrightjoint2.center);
    \draw[dashed] (concatstart2.center) to (concatrightjoint1.center);

    \draw (3.5, -3) node[output] (legendsymbol1) {$\boldsymbol{\oplus}$};
    \draw (3, -3.75) node[output] (legendsymbolstart) {};
    \draw (4.5, -3.75) node[output] (legendsymbolend) {};
    \draw (5, -3) node[output] (legend1) {concatenation};
    \draw (5, -3.75) node[output] (legend2) {disable backpropagation during training};
    \draw[arrowstyle, dashed] (legendsymbolstart.center) to (legendsymbolend.center);
\end{tikzpicture}

\caption[The architecture of the model.]{The architecture of the model. There are five separate modules of the entire model in total: \emph{beat}, \emph{quantization}, \emph{time signature}, \emph{key signature} and \emph{hand parts}.}
\end{figure}

\section{Score Generation}

The final score generation process synthesizes the source MIDI stream, incorporating pitch sequence, with the model's output. The resultant annotated MIDI encompasses essential information for visual score generation, including key/time signatures, which are not typically present in standard MIDI files.
